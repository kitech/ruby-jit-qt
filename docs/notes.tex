\section{TODO列表}


\section{默认参数值的传递表达}
\subsection{普通类型}
这种类型包括内置类型,int,char,QVariant支持的类。
这种可以直接使用QVariant表示,并且能够非常容易的检测。

\subsection{表达式参数值}
这种包括类似QChar(' '), QString(""),QFlags<???>(),
在FrontEngine中转换打包这个表达式,
传递到CompilerEngine进行编译相应的表达式,并传递到OperatorEngine中引用表达式结果。
表达式的结果应该在CompilerEngine中生成,因为底层不应该再调用上层。
不过这个表达式代码还是挺难生成的。

默认值表达式生成的执行逻辑优化:
1、目前是在FE中,检测表达式中的类类型,在AST查找时就明确生成这个类的实例,把这个类传递到IR生成类OE中使用。
这种实现没有灵活性,并且需要在后续的测试中不断加入新的默认值类型的处理。
2、现在,已经使用了更优化的方式,为默认值表达式生成一段IR代码,传递到IR生成类OE中使用。
一般默认值表达式不会太复杂,生成的IR代码也比较简单,也容易处理使用。
3、也许,后续还可以把这些默认值表达式缓存,把生成的IR代码也缓存,不需要每次使用的时候都重复处理了。

// 发现了点东西,拷贝IR指令时需要注意的。
/*
  可能就是api中说的,clone出来的Inst与原来的Inst有点不一样,没有Parent。如下所示:

  以下两名都能生成相同的一行指令,但前者生成的语句就会导致崩溃,后者生成的则无问题
  call void @_ZN6QFlagsIN2Qt10WindowTypeEEC2EMNS2_7PrivateEi(%class.QFlags* %toargx0, i32 -1)
  call void @_ZN6QFlagsIN2Qt10WindowTypeEEC2EMNS2_7PrivateEi(%class.QFlags* %toargx0, i32 -1)
*/


\subsection{void*类型}
这种是一个透明传递的QXxx*类型实例。

\section{模板方法的实例化}
\subsection{普通方法实例化}
\subsection{构造函数实例化}
默认情况下会生成Ctor\_Complete类型的实例。

\section{调用重载函数的查找确认}
查找最合适的调用函数。
这个应该在clang中有,不知道在哪?

基础原理,计算重载函数的参数匹配度,使用匹配度最高的那个。
匹配度计算,参数是否匹配。
1。名字匹配
2。对于构造,如果调用提供了参数,则忽略默认构造函数
3。函数的参数个数不能少于调用提供的参数个数。

经过这3步的过滤,效果还是比较好的。


\section{返回的处理}

\section{类属性赋值的处理}
lcd.value = 5
lcd.range = 0..99
估计需要考虑检测是否是类的属性类型的symbol,再转换成相应的set函数调用。
ruby会生成一个"interval="方法调用,有一个参数为赋值表达式的方法调用。
这里的=号,无论在ruby离interval多远,都会被连接到property名字上当作一个方法名。

\section{类重载操作符的处理}
像==、!=、>、<等重载的操作符号。

\section{qApp全局变量的处理}
qtbinding中实例这个变量的方式是,通过继承出来一个假的Application或者CoreApplication类,
在这个类的初始化时设置ruby全局变量$qApp。

这种方式,实际也没有太多的技巧,也许可以考虑更好的方式。
使用rb\_define\_virtual\_variable方式,在C部分查找到相应的qApp并返回。
这是使用的尽量在C模块中集中实现所有的绑定功能,不需要在ruby中再做绑定的补丁。

\section{预编译头文件AST的应用}
可用。
不完善,修改了其中一个头文件,则需要重新生成ast文件。
如果实时生成,则速度太慢,上秒级。
可以考虑使用增量式的AST生成。

\chapter{Ruby&Qt}
\section{Ruby nil}
nil传递到Ruby C API之后会是什么类型的值,怎么表达的。
nil传递到C时是int类型的0。
区间0..99传递到Ruby C API之后会是什么类型,怎么表达的。
即Range类型(内部表示为T\_STRUCT类型),需要扩展成两个参数:起始值与结束值。

\section{Ruby的层信号定义}
使用signals 'hit()', 'missed()', 'changed(int)'语法
处理过程:
在ruby继续类中,使用该语法定义
这个定义会生成一个signals(多维)哈希数组结合存储这些信号。
这个信号可以连接Qt原始类的slot,也可连接Ruby继承类的slot(自动检测,只要是变通方法即可)
可以使用Qt5::rbconnect方法,与Qt5::connectrb方法对应,表示信号是在ruby层定义的。
也许还需要一个Qt5::rbconnectrb方法,处理从ruby层的信号,连接到ruby层的slot。
还有一个默认的Qt5::QObject::connect,是使用qt层signal连接qt层的slot方法。
最后,几个连接模式全部完成后,把几种连接方式合并成为统一的通用方式。

在qtbindings中,也是使用类似的方式,但这些处理完全在ruby层实现的。
另外qtbindings中,slots需要显式指定,与signals一样。
而这里slots不需要显式指定,优先检测ruby层类是否有这个方法,如果有则优先使用,如果没有则使用qt层的slot方法。

\section{Ruby中继承Qt类的处理}

\section{Ruby的Object类已有方法的处理}
像display方法,如果在Qt类中还有这个方法,则无法正确的调用到Qt类的方法,
必须把Object类中的display方法undef掉,使用rb\_undef\_method函数。
类似的情况可能还有许多,但需要一个一个列出来。
这是还涉及到BasicObject类中的方法。
因为继承层次为BasicObject <- Object。

在ruby qtbinding项目中,它使用的是在ruby层重定义并覆盖这个方法的方式,
在qtruby4.rb文件中。
并且它使用的是覆盖整个Object类的方法,而不是针对某一个Qt类实例相应的修正。

\section{Ruby中new Qt类时,外带do |x| 块语法的处理}
怎么获取,获取之后怎么执行。
使用rb\_block\_given\_p()判断是否外带block语法,如果有,则直接rb\_yield(self)执行。

\section{ruby中动态注册Qt类}
目前的情况,先使用rb\_define\_class等来声明类,使用rb\_define\_method等声明方法。
这种需要预先列出所有的qt类,比较麻烦,也不够通用,不够有效率(启动注册上千个类)。
但是,在不预先注册qt类的情况下,ruby会把这个当作constant处理,
如Qt5::QStringAbc.new()语句,会执行到x\_Qt\_Constant\_missing函数中。
在这个函数中,动态注册该Qt类,并返回类对应的VALUE类型的值。

\section{clang与gcc兼容性问题}
\subsection{/usr/include/string.h:82:1: error: unknown type name '\_\_extern\_always\_inline'}
在使用clang编译时报错。
一种解决方式是使用-O0优化参数,能够避免这个报错。

\section{Qt库调试}
可以另行编译出Qt共享库,使用设置LD\_LIBRARY\_PATH方式,
用新编译出的库替换系统自带的库文件,这样可以比较方便的在Qt中插入代码调试。

\section{Qt类实例解引}
目前是存储在void*中,在qDebug()的时候需要根据实例类型执行解引,这就需要动态获取类是否能够支持的类型。
第一种方式是使用AST树进行查找,按照正常的调用函数方式。
第二种是使用dlopen/dlsym方式,这种方式也不会比第1种方式快,需要打开共享库文件,有磁盘操作。
第三种是把所有支持qDebug操作的类罗列出来,并批量写到代码中。

3:

\chapter{C++}
nm -s libhandby.so 中,符号为'U'的表示什么意思呢?
执行的时候报错:undefined symbol: \_ZN9SingletonI15QtObjectManagerE6m\_instE
这个'U'应该就是undefined的意思。

C++中的friend友元不能继承。

\section{参考}
ROOT/cling
llvm/vmkit
Graal/Truffle

